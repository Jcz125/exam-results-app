% \documentclass[12pt]{article}
% \usepackage[french]{babel}
% \usepackage{natbib}
% \usepackage[utf8]{inputenc}
% \usepackage[T1]{fontenc}
% \usepackage{tikz}
% \usepackage{amsmath}
% \usepackage{graphics}
% \usepackage{graphicx}
% \usepackage{url}
% \usepackage{psfrag}
% \usepackage{fancyhdr}
% \usepackage{vmargin}
% \usepackage[backend=biber]{biblatex}
% \usepackage{csquotes}
% \usepackage[hidelinks]{hyperref}
% \usepackage{enumitem}

% \pagestyle{fancy}


% \begin{document}
\subsubsection*{\large{Réunion d'équipe du 25 avril 2021}}
    \addcontentsline{toc}{subsubsection}{Réunion d'équipe du 25 avril 2021}
\begin{center}
\begin{tabular}{| l | l || c | c |}
    \hline
    Membres présents & Membres absents & Durée & Lieu \\
    \hline
    Maël SAILLOT & & & \\ Céline ZHANG & & 1h40 & Discord \\ Ahmed ZIANI & & & \\
    \hline
\end{tabular}
\end{center}

\subsubsection*{Ordre du jour}
\begin{enumerate}
    \item Avancement des tâches
    \item Discussion générale et amélioration de la structure actuelle
    \item Prochaines tâches
\end{enumerate}

\subsubsection*{Avancement des tâches}
\paragraph{Maël SAILLOT} a ajouté les tables, les champs manquants, a créé les fichiers \textsf{createdb.sql} et \textsf{schemadb.txt} pour le partage.
\paragraph{Céline ZHANG} a revu les codes du schéma de \textsf{dbdiagram.io}, a commencé à remplir le fichier \textsf{createdb.sql}.
\paragraph{Ahmed ZIANI} a relu les travaux de l'équipe.

\subsubsection*{Discussion générale et amélioration de la structure actuelle}
Nous avons relu chaque ligne de schéma, nous avons fixé les contraintes nécessaires sur certains champs, et corrigé les erreurs. Les codes postaux doivent être en \textsf{TEXT} car les adresses étrangères ont des lettres dans leur code postal, idem pour les numéros particulier comme \textsf{+33(0)}.

\subsubsection*{Prochaines tâches}
L'équipe devra compléter le fichier \textsf{createdb.sql}, ajouter les contraintes, vérifier le respect des formes normalisés et les contraintes (\textsl{review}), et se renseigner sur l'écriture d'un script en \textsf{Python} qui permettra de remplir la base de données de manière automatique.


\paragraph{\emph{TO-DO LIST}}
\begin{itemize}
    \item Remplir le fichier \textsf{createdb.sql} et tester l'exécution de celui-ci
    \item Ajouter les \textsf{CHECK} nécessaires (voir le canal \textsl{\#todo} sur Discord)
    \item Partager fichier \textsf{.py} sur le dépôt \texttt{git}
    \item Se documenter sur l'écriture du script en \textsf{.py}
    \item Écrire un exemple de fonction dans le script pour l'import des données
    
\end{itemize}

\emph{Prochaine réunion : 29/04/2021}\\

% \end{document}