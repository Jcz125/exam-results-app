% \documentclass[12pt]{article}
% \usepackage[french]{babel}
% \usepackage{natbib}
% \usepackage[utf8]{inputenc}
% \usepackage[T1]{fontenc}
% \usepackage{tikz}
% \usepackage{amsmath}
% \usepackage{graphics}
% \usepackage{graphicx}
% \usepackage{url}
% \usepackage{psfrag}
% \usepackage{fancyhdr}
% \usepackage{vmargin}
% \usepackage[backend=biber]{biblatex}
% \usepackage{csquotes}
% \usepackage[hidelinks]{hyperref}
% \usepackage{enumitem}

% \pagestyle{fancy}


% \begin{document}
\subsubsection*{\large{Réunion d'équipe du 11 avril 2021}}
    \addcontentsline{toc}{subsubsection}{Réunion d'équipe du 11 avril 2021}
\begin{center}
\begin{tabular}{| l | l || c | c |}
    \hline
    Membres présents & Membres absents & Durée & Lieu \\
    \hline
    Maël SAILLOT & & & \\ Céline ZHANG & & 1h & Discord \\ Ahmed ZIANI & & & \\
    \hline
\end{tabular}
\end{center}

\subsubsection*{Ordre du jour}
\begin{enumerate}
    \item Avancement des tâches
    \item Discussion générale des avis sur la structure actuelle
    \item Optimisation des solutions proposées
    \item Prochaines tâches
\end{enumerate}

\subsubsection*{Avancement des tâches}
\paragraph{Maël SAILLOT} a écrit dans le \textsf{dbdiagram.io} la première version de la structure de la base de données et l'a partagé aux autres membres.
\paragraph{Céline ZHANG} a complété une partie des champs manquants dans les tables et ajouté une partie des tables manquantes.
\paragraph{Ahmed ZIANI} a proposé une solution en faisant des tables \texttt{admissible\_XX}, \texttt{admis\_XX} pour chaque filière.

\subsubsection*{Discussion générale des avis sur la structure actuelle}
La proposition des tables \texttt{admissible\_XX}, \texttt{admis\_XX} pour chaque filière a finalement été rejeté, puisqu'il y aurait redonnance des données, par ailleurs, il se trouvait déjà dans le champ \texttt{type} (qui n'avait pas été explicité, d'où le malentendu avec les tables \texttt{admissible\_XX}, \texttt{admis\_XX}).

\subsubsection*{Optimisation des solutions proposées}
Nous avons vérifié les relations entre les tables, et corrigé quelques imprecision des champs. Nous avons rajouté des notes et restrictions pour certains champs. Après discussion, nous avons choisi de mettre une partie des info candidat dans candidat, et une autre partie certainement dans une autre table, elles seront liées par une relation 1 1.

\subsubsection*{Prochaines tâches}
L'équipe devra compléter les tables, et écrire les tables restantes sur \textsf{dbdiagram.io}. Ensuite, chaque membre fera 3 à 4 tables en SQLite, et mettra leur code sur un fichier \textsf{.sql} (ou \textsf{.txt}) partagé sur le \textsf{git}. Chacun sera libre de choisir les tables qu'il fera, mais il devra indiquer sur le Trello son choix et \textsl{push} son travail pour tenir au courant les autres membres. Une description détaillée de ce qui est à faire se trouve sur le Discord de l'équipe, dans le canal \textsl{\#todo}.


\paragraph{\emph{TO-DO LIST}}
\begin{itemize}
    \item Finir de mettre les tables manquantes (voir le canal \textsl{\#todo} sur le Discord de l'équipe)
    \item Finir de remplir les champs des tables (voir le canal \textsl{\#todo} sur Discord)
    \item Implémenter ces tables en SQLite et partager les codes sur le dépôt \texttt{git}
    \item Mettre à jour le Trello
    
\end{itemize}

\emph{Prochaine réunion : 25/04/2021}\\

% \end{document}